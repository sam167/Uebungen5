\documentclass[11pt,a4paper]{article}
\usepackage[german]{babel}
 \date{02. November 2014}
 \author{Samantha Novak}
 \title{CS102 \LaTeX \, \"Ubung}
 
 \begin{document}
\maketitle
 \section{Das ist der erste Abschitt}
 Tiere sind eukaryotische Lebewesen, die ihre Energie nicht durch Photosynthese gewinnen und Sauerstoff zur Atmung benötigen, aber keine Pilze sind.
 \section{Tabelle}
 Unsere wichtigsten Daten finden Sie in Tabelle 1.
\begin{table}[h]
\centering
\begin{tabular}{c|c|c|cl}
          & Punkte erhalten & Punkte m"oglich & \%  &  \\ \cline{1-4}
Aufgabe 1 & 7               & 6             & 0.5 &  \\
Aufgabe 2 & 8               & 2              & 1   &  \\
Aufgabe 3 & 4               & 2              & 1   & 
\end{tabular}
\caption{Diese Tabelle kann auch andere Werte beinhalten}
\label{tab:formen}
\end{table}
\section{Formeln}
\subsection{Pythagoras}
Der Satz des Pythagoras errechnet sich wie folgt: $ a^2 + b^2 = c^2 $. Daraus k"onnen wir die L"ange der Hypothenuse wie folgt berechnen:
$ c=\sqrt{a^{2}+b^{2}} $
\subsection{Summen}
Wir k"onnen auch die Formel f"ur eine Summe angeben:
\begin{equation}
s=\sum_{i=1}^n i = \frac{n *\ (n+1)}{2}
\end{equation}
 
 \end{document}
 ich war hier
 
 Gruß Luisa
 

 
